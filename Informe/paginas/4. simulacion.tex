\section{Simulación}
A continuación se mostrarán las diferentes simulaciones propuestas por el enunciado, junto con el análisis de los resultados obtenidos. Es importante mencionar que la simulación se retrasó en 100 [ms] respecto a los tiempos que se solicitaron. Esto fue decidido para que la simulación alcance el estado estacionario antes de que comiencen las perturbaciones.

%==================================== e) ====================================
\subsection{\textit{Escenario con tres perturbaciones de carga.}}
%==================================== e) ====================================

\begin{table}[H]
\centering
\begin{tabular}{|c|c|}
\hline
\textbf{Inicialización del sistema} & \textbf{Resistencia de carga} \\
\hline
$0 \leq t < 2$ (seg.) & $R_L = 10\,\Omega$ \\
\hline
$2 \leq t < 5$ (seg.) & $R_L = 3.33\,\Omega$ \\
\hline
$5 \leq t < 8$ (seg.) & $R_L = 2.3\,\Omega$ \\
\hline
$8 \leq t < 12$ (seg.) & La carga se desconecta, $R_L \rightarrow \infty$ \\
\hline
\end{tabular}
\caption{Variación de la resistencia de carga durante la simulación.}
\label{tab:e}
\end{table}

\begin{figure}[H]
    \centering
    \includegraphics[width=1\linewidth]{img/simulación/e_corrientes.png}
    \caption{Corrientes del sistema.}
    \label{fig:e_corrientes}
\end{figure}

\begin{figure}[H]
    \centering
    \includegraphics[width=1\linewidth]{img/simulación/e_voltajes_zoom.png}
    \caption{Voltajes con zoom.}
    \label{fig:e_voltajes_zoom}
\end{figure}

\begin{figure}[H]
    \centering
    \includegraphics[width=1\linewidth]{img/simulación/e_voltajes.png}
    \caption{Voltajes del sistema.}
    \label{fig:e_voltajes}
\end{figure}

%==================================== f) ====================================
\subsection{\textit{Simulación con dos cambios en la carga}}
%==================================== f) ====================================

\begin{table}[H]
    \centering
    \begin{tabular}{|c|c|}
    \hline
    \textbf{Inicialización del sistema} & \textbf{Resistencia de carga} \\
    \hline
    $0 \leq t < 2$ (seg.) & $R_L = 10\,\Omega$ \\
    \hline
    $2 \leq t < 5$ (seg.) & Resistencia de carga total de $R_L = 2.3\,\Omega$ \\
    \hline
    $5 \leq t < 8$ (seg.) & La carga total se desconecta y $R_L \rightarrow \infty$ \\
    \hline
\end{tabular}
\caption{Escenario con dos perturbaciones de carga.}
\label{tab:f}
\end{table}


\begin{figure}[H]
    \centering
    \includegraphics[width=1\linewidth]{img/simulación/f_corriente_1.98-2.1.png}
    \caption{Corriente entre 1.98 y 2.1 segundos.}
    \label{fig:f_corriente_1.98-2.1}
\end{figure}

\begin{figure}[H]
    \centering
    \includegraphics[width=1\linewidth]{img/simulación/f_corriente_4.98-5.1.png}
    \caption{Corriente entre 4.98 y 5.1 segundos.}
\label{fig:f_corriente_4.98-5.1}
\end{figure}

\begin{figure}[H]
    \centering
    \includegraphics[width=1\linewidth]{img/simulación/f_maximosobrepaso.png}
    \caption{Máximo sobrepaso del sistema.}
\label{fig:f_maximosobrepaso}
\end{figure}

\begin{figure}[H]
    \centering
    \includegraphics[width=1\linewidth]{img/simulación/f_tiempoestablecimientofuelcell.png}
    \caption{Tiempo de establecimiento de la celda de combustible.}
    \label{fig:f_tiempoestablecimientofuelcell}
\end{figure}

%==================================== g) ====================================


\begin{figure}[H]
    \centering
    \includegraphics[width=1\linewidth]{img/simulación/g_volt1.98-2.1.png}
    \caption{Voltaje entre 1.98 y 2.1 segundos.}
    \label{fig:g_volt1.98-2.1}
\end{figure}

\begin{figure}[H]
    \centering
    \includegraphics[width=1\linewidth]{img/simulación/g_volt4.98-5.1.png}
    \caption{Voltaje entre 4.98 y 5.1 segundos.}
    \label{fig:g_volt4.98-5.1}
\end{figure}

%==================================== h) ====================================
\subsection{\textit{Sistema sin Anti-Winding Up}}
%==================================== h) ====================================

\begin{figure}[H]
    \centering
    \includegraphics[width=1\linewidth]{img/simulación/h_corrientes.png}
    \caption{Corrientes del sistema en otra condición.}
    \label{fig:h_corrientes}
\end{figure}

\begin{figure}[H]
    \centering
    \includegraphics[width=1\linewidth]{img/simulación/h_voltajes.png}
    \caption{Voltajes del sistema en otra condición.}
    \label{fig:h_voltajes}
\end{figure}

%==================================== i) ====================================
\subsection{\textit{Sistema con feedforward}}
%==================================== i) ====================================

\begin{figure}[H]
    \centering
    \includegraphics[width=1\linewidth]{img/simulación/i_corrientes_conAWU.png}
    \caption{Corrientes con Anti-WindUp.}
    \label{fig:i_corrientes_conAWU}
\end{figure}

\begin{figure}[H]
    \centering
    \includegraphics[width=1\linewidth]{img/simulación/i_corrientes_sinAWU.png}
    \caption{Corrientes sin Anti-WindUp.}
    \label{fig:i_corrientes_sinAWU}
\end{figure}

\begin{figure}[H]
    \centering
    \includegraphics[width=1\linewidth]{img/simulación/i_voltajes_conAWU.png}
    \caption{Voltajes con Anti-WindUp.}
    \label{fig:i_voltajes_conAWU}
\end{figure}

\begin{figure}[H]
    \centering
    \includegraphics[width=1\linewidth]{img/simulación/i_voltajes_sinAWU.png}
    \caption{Voltajes sin Anti-WindUp.}
    \label{fig:i_voltajes_sinAWU}
\end{figure}

\newpage